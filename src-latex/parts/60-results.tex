% !TeX root = ../main-paper.tex
\section{Results}

\subsection{Experiment 1: Accuracy vs NER approach vs training set size}

Qualitative results
\textbf{TODO random samples of results + selection of failure cases}


Quantitative results: table + graph ideally (with training set size info)

\begin{verbatim}
Accuracy
|                             ..o   
|                ..o       o/       
|            ./`          /`          
|        ./o`      .o   o            
|     ./`        .`                    One curve per approach
|    o       o/`                  o  
|          .`                  ./    
|       o`                    /      
|                         .o        
|                       .`           
|                      o            
|                                   
+------------------------------   Training set size
\end{verbatim}
                                        

\subsection{Experiment 2: sensibility to training on noisy data}
Idea : 

Measure : M


\subsection{Experiment 3: Accuracy vs NER approach vs OCR noise}

Qualitative results
\textbf{TODO random samples of results + selection of failure cases}

Quantitative results: table + graph ideally (with OCR noise, same format as previous)

Opt. use synthetic text perturbation as well? Maybe not interesting and too artificial if we have access to 2+ OCR systems.
(original OCR from BNF, Tesseract 4, Pero OCR…)


\subsection{Discussion}
Interesting points to discuss:
\begin{itemize}
    \item can we train on noisy data? (without manual OCR correction?) => future work? cf Pero OCR training procedure?
    \item do we need better OCR systems or better post-correction techniques (if NER is reliable enough)?
    \item Construction of the lexicon and associated cost
\end{itemize}
