% !TeX root = ../main-paper.tex
\section{Named Entity Recognition in Historical Texts}

The directory processing pipeline presented in \cite{bell2020automated} to locate old gas stations using mid-20th century directories includes an OCR step, done with Tesseract, and a NER step to identify company names and addresses within the extracted text, performed using rules built as regular expressions. %They look for lexical and layout patterns to retrieve portions of text that refer to street names, numbers, and company names.

Many approaches have been designed to recognize named entities, ranging from handcrafted rules to supervised approaches \cite{nadeau2007}. Rule based approaches look for portions of text that match patterns based on word characteristics such as their case, whether they are digits or not, their part-of-speech tags, etc. These patterns can be fully handcrafted like in \cite{bell2020automated} or generated by data mining approaches like in \cite{nouvel2011}. Rule based approaches can also combine a set of patterns with a set of dictionaries (gazetteers, author lists, etc.) that help recognizing named entities when an exhaustive lexicon of searched entities is available \cite{mansouri2008,maurel2011}. For example, the CasEN\footnote{\url{https://tln.lifat.univ-tours.fr/version-francaise/ressources/casen}} transducer cascade leverages syntatic-lexical patterns and dictionnaries to recognize Named Entities in French texts. Such kind of approaches achieve very good results when applied to specialized domain corpus - like directories - and when exhaustive lexicon are available, but at high system engineering cost \cite{nadeau2007}. 

Supervised approaches include both approaches implementing supervised learning algorithms with careful text feature engineering, and deep learning based approaches which automatically build their own features to classify words into one category of named entities or another. In recent years, the latter have grown significantly, surpassing the performance of rule-based systems and feature-based supervised approaches and establishing new baselines \cite{li2020}.


NER for historical documents given our constraints