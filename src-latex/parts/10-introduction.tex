% !TeX root = ../main-paper.tex
\section{Introduction}

\textbf{TODO sample (simplified) pipeline of the information extraction process}
\textbf{and/or}
\textbf{TODO illustration a sample page (full image, large crop, entry crop with highlighted entities)}

% General context + scope limitation
NER is a fundamental step of the information extraction pipeline when processing historical listing, like directories for instance.
OCRed text is not sufficient to build a high level, semantic view of the collection under study;
indeed, being able to properly tag text tokens unlocks the ability to relate entities,
and provide colleagues from digital humanities with useful databases.

% Problematic
NER requires training data, and we have little training data for historical documents.
OCR is also noisy.
How can we extract reliable elements in this context?

% Contributions
\begin{enumerate}
    \item review state of the art regarding NER for historical documents (section 2)
    \item dataset (section 3)
    \item list usable techniques (more to evangelize the DAR community) (section 4)
    \item compare their performance under various training set sizes (thus annotation costs) and OCR noise levels (thus OCR choices/luck/availability) (sec. 5 and 6)
    \item suggest smart way to leverage side knowledge like lexicons (FIXME iif this works well, maybe a conclusion instead)
\end{enumerate}


