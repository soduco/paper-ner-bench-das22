% !TeX root = ../main-paper.tex
\section{Dataset}

In this work, we  aim  at  producing  structured  spatio-temporal  data  from the information contained in the 19th century Parisian trade directories.
To do this, it is therefore necessary to extract the text from the scans of the directories to be processed and to identify the entries in the various indexing lists used.
Then, in each of these entries, the named entities they contain have to be extracted to produce structured spatio-temporal data representing the evolution over time of the people and businesses listed in these directories, their descriptions and their locations.
Both of these extraction tasks require data to train and evaluate the OCR and NER approaches identified as potentially relevant: Tesseract and PERO OCR for the OCR task and Spacy, and CamemBERT - pre-trained or fine-tuned - neural models for the NER task.

\subsection{The 19th century Parisian trade directories}

\begin{table}[h!]
\resizebox{\textwidth}{!}{%
\begin{tabular}{|l|l|l|l|l|}
\hline
\multicolumn{1}{|c|}{\textbf{Directory publisher or name}} & \multicolumn{1}{c|}{\textbf{Year}} & \multicolumn{1}{c|}{\textbf{\# Col}} & \multicolumn{1}{c|}{\textbf{Index}} & \multicolumn{1}{c|}{\textbf{Entry structure}}                                                                                                       
\\ \hline
Favre \& Duchesne & 1798 & 1 & by activity & {N [A], SN, SNUM — SEC.} 
\\ \hline
Duverneuil et La Tynna & 1801 & 1 & \begin{tabular}[c]{@{}l@{}}by activity\\ by name\end{tabular} & {N, SN, SNUM. SEC.} 
\\ \hline
{Notables communaux de la Seine} & 1801 & 1 & by name & {N, [A,] SN[, SNUM].}
\\ \hline
Duverneuil et La Tynna & 1805 &	2 &	\begin{tabular}[c]{@{}l@{}}by activity\\ by name\end{tabular} & \begin{tabular}[c]{@{}l@{}}N [(F)], SN, SNUM. — SEC.\\ N [(C)], [(A),] SN, SNUM. — SEC.\end{tabular}
\\ \hline
Duverneuil et La Tynna & 1806 & 2 & \begin{tabular}[c]{@{}l@{}}by activity\\ by name\end{tabular} & \begin{tabular}[c]{@{}l@{}} N [(F)], SN, SNUM.\\ N [(C)], SN, SNUM.[ — SEC.]\end{tabular}.
\\ \hline
La Tynna & 1813	& 2	& \begin{tabular}[c]{@{}l@{}}by activity\\ by name\end{tabular} & \begin{tabular}[c]{@{}l@{}} N,[A,] SN[, SNUM].\\ N, [(T)] A, SN[, SNUM] [, D].\end{tabular}.
\\ \hline
Panckoucke Comm Dulac & 1820 & 1 & by name & N [(F)], A, [(T),] SN, SNUM[, P] .
\\ \hline
Panckoucke hab Dulac & 1820	& 1 & by name & N [(T)], [A,] SN, SNUM [, D].
\\ \hline
Bottin1 & 1820 & 2	& \begin{tabular}[c]{@{}l@{}}by activity\\ by name\end{tabular} & \begin{tabular}[c]{@{}l@{}} N [(F \textbar C)][T], [P], SN, SNUM.\\N [(F \textbar C)][T], A, SN, SNUM. CR\end{tabular}.
\\ \hline
Bottin1 & 1827 & 2	& \begin{tabular}[c]{@{}l@{}}by activity\\ by name\end{tabular} & \begin{tabular}[c]{@{}l@{}}N [(F \textbar C)][T], [P], SN, SNUM.\\ N [(F \textbar C)][T], A, SN, SNUM.\end{tabular}.
\\ \hline
Deflandre & 1828 & 2 & \begin{tabular}[c]{@{}l@{}}by activity\\ by name\end{tabular} & \begin{tabular}[c]{@{}l@{}} N [(F \textbar C)][T], [P], SN, SNUM. \\N [(F \textbar C)][T], A, SN, SNUM.\end{tabular}.
\\ \hline
Deflandre & 1829 & 2 & \begin{tabular}[c]{@{}l@{}}by activity\\ by name\end{tabular} & \begin{tabular}[c]{@{}l@{}}N [(F \textbar C)][T], [P], SN, SNUM.\\ N [(F \textbar C)][T], A, SN, SNUM.\end{tabular}.
\\ \hline 
Bottin1 & 1837 & 2	& \begin{tabular}[c]{@{}l@{}}by activity\\ by name\end{tabular} & \begin{tabular}[c]{@{}l@{}}N [(F \textbar C)][T], [P], SN, SNUM.\\ N [(F \textbar C)][T], [A], SN, SNUM.\end{tabular}.
\\ \hline
Cambon almgene & 1841 & 2 &	\begin{tabular}[c]{@{}l@{}}by activity\\ by name\end{tabular} & \begin{tabular}[c]{@{}l@{}}N [(F \textbar C)][T], [P], SN, SNUM.\\ N [(F \textbar C)][T], A, SN, SNUM.\end{tabular}.
\\ \hline
Didot & 1841a &	\begin{tabular}[c]{@{}l@{}}3\\ 4\end{tabular} &	\begin{tabular}[c]{@{}l@{}}by activity \\ by name \end{tabular} & 	\begin{tabular}[c]{@{}l@{}}N [(F \textbar C)][T], [P], SN, SNUM. \\ N [(F \textbar C)][T], [A], SN, SNUM. \end{tabular}
\\ \hline
Didot & 1851a & \begin{tabular}[c]{@{}l@{}l@{}}4\\ 4\\ 5\end{tabular} & \begin{tabular}[c]{@{}l@{}l@{}}by activity\\ by name\\ by street name\end{tabular} & \begin{tabular}[c]{@{}l@{}l@{}}N [(F \textbar C)][T], [P], SN, SNUM.\\ N [(F \textbar C)][T], [A], SN, SNUM.\\SNUM N [(F \textbar C)][T], [A].\end{tabular}
\\ \hline
Didot & 1854a & \begin{tabular}[c]{@{}l@{}l@{}}4\\ 3\\ 5\end{tabular} &	\begin{tabular}[c]{@{}l@{}l@{}} by activity \\ by name \\ by street name\end{tabular} & \begin{tabular}[c]{@{}l@{}l@{}} N [(F \textbar C)], [P], SN, SNUM [,T].\\ N [(F \textbar C)][T], [A], SN, SNUM.\\ SNUM N [(F \textbar C)][T], [A].\end{tabular}
\\ \hline
Bottin3 & 1854a & \begin{tabular}[c]{@{}l@{}l@{}}3\\ 2\\ 4\end{tabular} & \begin{tabular}[c]{@{}l@{}l@{}} by activity \\ by name \\ by street name\end{tabular} & \begin{tabular}[c]{@{}l@{}l@{}}N [(F \textbar C)], [P], SN, SNUM [,T].\\ N [(F \textbar C)][T], [A], SN, SNUM.\\ SNUM N [(F \textbar C)], [A] [,T].\end{tabular}
\\ \hline
DidotBottin & 1860a	& 3	& by name & N [(F \textbar C)][T], [A], SN, SNUM.
\\ \hline
DidotBottin & 1861a	& 3	& by name & N [(F \textbar C)][T], [A], SN, SNUM.
\\ \hline
\end{tabular}%
}
\caption{The directories used to build our dataset. Entry structure is described as follows: N - entry name (person or business); F - first name; T - honorary title; C - civility; A - activity; P - precision; SN - street name; SNUM - street number; SEC - section; D - district; CR - cross-reference}
\label{tab:my-table}
\end{table}

\subsection{Groundtruth data extraction}

\subsection{Assessing the groundtruth quality}

Our dataset is thus made of excerpts 

TODO interesting points to detail:
\begin{itemize}
    \item origin
    \item illustration (maybe in first figure)
    \item various numbers (of the original data and of the sample we extracted) : number of entries per pages, variety of directory entries, \dots
    \item statistics about entry lengths (histogram of number of char / tokens par entry)
    \item entities we considered and how we annotated them (how we assessed gold's quality)
    \item statistics on the distribution of entities across directories and their arrangement within entries
    \item OCR data (automated systems and clean text + how we created the "clean" text + whether/how we assessed clean text's quality)
    \item Measure inter-annotator agreement ? 
    \item Explain charset projection
    \item \dots
\end{itemize}
