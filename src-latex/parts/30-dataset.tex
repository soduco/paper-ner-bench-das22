% !TeX root = ../main-paper.tex
\section{Dataset}

%In this work, we  aim  at  producing  structured  spatio-temporal  data  from the information contained in the XIX\textsuperscript{th} century Parisian trade directories.
%To do this, it is therefore necessary to extract the text from the scans of the directories to be processed and to identify the entries in the various indexing lists used.
%Then, in each of these entries, the named entities they contain have to be extracted to produce structured spatio-temporal data representing the evolution over time of the people and businesses listed in these directories, their descriptions and their locations.
%Both of these extraction tasks require data to train and evaluate the OCR and NER approaches identified as potentially relevant: Tesseract and PERO OCR for the OCR task and Spacy and CamemBERT - pre-trained or fine-tuned - neural models for the NER task.

\subsection{The  XIX\textsuperscript{th} century Parisian trade directories}

The directories to be processed are stored in different libraries in Paris\footnote{\textcolor{green}{Citer les bibliothèques sources?}}. They have been scanned independently and with various levels of quality. Moreover, they cover a wide period and have been produced by different publishers and printers. Therefore, their contents, indexes, layouts, methods of printing, etc. may vary a lot from one directory to another (see Figure \ref{fig:directories}).


\begin{figure}[htb!]
	   \center{\includegraphics[width=0.9\textwidth]
	       {./images/DirectoryExcerpts2.png}}
	  \caption{\label{fig:directories} Examples of directory layouts and contents: 1) Duverneuil et La Tynna 1806 - index by name; 2) Deflandre 1828 - index by activity ; 3) Bottin 1851 –  index by street name}
\end{figure}

For training as testing purposes, our groundtruth dataset must be as representative as possible of the diversity of the available directories. The Table \ref{tab:directories} lists the directories selected, their layout, the structure of their entries, and the number of annotated entries according to the considered index.

For example, the Didot directory of 1854 is organised according to three indexes: by name, by activity, and by street name. The structure of the following entry is characteristic of its index by name: "Batton (D.-A.) \scalerel*{\includegraphics{./images/LH.png}}{H}, professeur au Conservatoire de musique et de déclamation, Saint-Georges, 47." This entry begins with a person's name, followed by the initials of their first name in parentheses.The symbol indicates that this person was awarded the Légion d'Honneur. The following part of this entry represents the person's activity,i.e. their profession, or social status, here "professor at the Conservatory of music and declamation". The street name and number where the person lives or carries out their activity are written at the end of the entry.

These pieces of information constituting the structure of the entries are the basic data that we want to extract, deduplicate and structure in order to build a spatio-temporal database of XIX\textsuperscript{th} century Parisian inhabitants and businesses. With the exception of some potentially wordy activity descriptions, they correspond to named entities. However, as Table \ref{tab:directories} shows, while most entries contain the same types of named entities, their order and the way they are written vary from one directory or index to another. To provide examples of each entry structure, pages from each type have thus been annotated\footnote{In this work, the street name index entries have not been included as they are very different from the others. The strategy to be adopted to process them without loss of quality for the NER model will be the subject of future work.}.

\begin{table}[h!]
\resizebox{\textwidth}{!}{%
\begin{tabular}{|l|l|l|l|l|l|}
\hline
\multicolumn{1}{|c|}{\textbf{Directory publisher or name}} & \multicolumn{1}{c|}{\textbf{Year}} & \multicolumn{1}{c|}{\textbf{\# Col}} & \multicolumn{1}{c|}{\textbf{Index}} & \multicolumn{1}{c|}{\textbf{Entry structure}} & \multicolumn{1}{c|}{\textbf{\# Ann}}                                                                                                      
\\ \hline
Favre \& Duchesne & 1798 & 1 & by activity & {N [A], SN, SNUM - SEC.} & 233
\\ \hline
Duverneuil et La Tynna & 1801 & 1 & \begin{tabular}[c]{@{}l@{}}by activity\\ by name\end{tabular} & {N, SN, SNUM. SEC.} & \begin{tabular}[c]{@{}l@{}} 136 \\ 103\end{tabular}
\\ \hline
{Notables communaux de la Seine} & 1801 & 1 & by name & {N, [A,] SN[, SNUM].} & 92
\\ \hline
Duverneuil et La Tynna & 1805 &	2 &	\begin{tabular}[c]{@{}l@{}}by activity\\ by name\end{tabular} & \begin{tabular}[c]{@{}l@{}}N [(F)], SN, SNUM. - SEC.\\ N [(C)], [(A),] SN, SNUM. - SEC.\end{tabular} & \begin{tabular}[c]{@{}l@{}} 155 \\ 221 \end{tabular}
\\ \hline
Duverneuil et La Tynna & 1806 & 2 & \begin{tabular}[c]{@{}l@{}}by activity\\ by name\end{tabular} & \begin{tabular}[c]{@{}l@{}} N [(F)], SN, SNUM.\\ N [(C)], SN, SNUM.[ - SEC.]\end{tabular}. & \begin{tabular}[c]{@{}l@{}} 186\\ 0\end{tabular}
\\ \hline
La Tynna & 1813	& 2	& \begin{tabular}[c]{@{}l@{}}by activity\\ by name\end{tabular} & \begin{tabular}[c]{@{}l@{}} N,[A,] SN[, SNUM].\\ N, [(T)] A, SN[, SNUM] [, D].\end{tabular} & \begin{tabular}[c]{@{}l@{}} 135 \\ 177 \end{tabular}
\\ \hline
Panckoucke Commerces (Dulac) & 1820 & 1 & by name & N [(F)], A, [(T),] SN, SNUM[, P]. & 0
\\ \hline
Panckoucke Habitants (Dulac) & 1820	& 1 & by name & N [(T)], [A,] SN, SNUM [, D]. & 0
\\ \hline
Bottin (serie 1: 1819-1838) & 1820 & 2	& \begin{tabular}[c]{@{}l@{}}by activity\\ by name\end{tabular} & \begin{tabular}[c]{@{}l@{}} N [(F \textbar C)][T], [P], SN, SNUM.\\N [(F \textbar C)][T], A, SN, SNUM. CR\end{tabular} & \begin{tabular}[c]{@{}l@{}} 128 \\ 151 \end{tabular}
\\ \hline
Bottin (serie 1: 1819-1838) & 1827 & 2	& \begin{tabular}[c]{@{}l@{}}by activity\\ by name\end{tabular} & \begin{tabular}[c]{@{}l@{}}N [(F \textbar C)][T], [P], SN, SNUM.\\ N [(F \textbar C)][T], A, SN, SNUM.\end{tabular} & \begin{tabular}[c]{@{}l@{}} 228 \\ 285 \end{tabular}
\\ \hline
Deflandre & 1828 & 2 & \begin{tabular}[c]{@{}l@{}}by activity\\ by name\end{tabular} & \begin{tabular}[c]{@{}l@{}} N [(F \textbar C)][T], [P], SN, SNUM. \\N [(F \textbar C)][T], A, SN, SNUM.\end{tabular} & \begin{tabular}[c]{@{}l@{}}115\\ 229\end{tabular}
\\ \hline
Deflandre & 1829 & 2 & \begin{tabular}[c]{@{}l@{}}by activity\\ by name\end{tabular} & \begin{tabular}[c]{@{}l@{}}N [(F \textbar C)][T], [P], SN, SNUM.\\ N [(F \textbar C)][T], A, SN, SNUM.\end{tabular} & \begin{tabular}[c]{@{}l@{}}177\\ 236\end{tabular}
\\ \hline 
Bottin (serie 1: 1819-1838) & 1837 & 2	& \begin{tabular}[c]{@{}l@{}}by activity\\ by name\end{tabular} & \begin{tabular}[c]{@{}l@{}}N [(F \textbar C)][T], [P], SN, SNUM.\\ N [(F \textbar C)][T], [A], SN, SNUM.\end{tabular} & \begin{tabular}[c]{@{}l@{}}158\\ 557\end{tabular}
\\ \hline
Cambon - Almanach général & 1841 & 2 &	\begin{tabular}[c]{@{}l@{}}by activity\\ by name\end{tabular} & \begin{tabular}[c]{@{}l@{}}N [(F \textbar C)][T], [P], SN, SNUM.\\ N [(F \textbar C)][T], A, SN, SNUM.\end{tabular} & \begin{tabular}[c]{@{}l@{}}182\\ 486\end{tabular}
\\ \hline
Didot & 1841a &	\begin{tabular}[c]{@{}l@{}}3\\ 4\end{tabular} &	\begin{tabular}[c]{@{}l@{}}by activity \\ by name \end{tabular} & 	\begin{tabular}[c]{@{}l@{}}N [(F \textbar C)][T], [P], SN, SNUM. \\ N [(F \textbar C)][T], [A], SN, SNUM. \end{tabular} & \begin{tabular}[c]{@{}l@{}}246\\ 843\end{tabular}
\\ \hline
Didot & 1851a & \begin{tabular}[c]{@{}l@{}l@{}}4\\ 4\\ 5\end{tabular} & \begin{tabular}[c]{@{}l@{}l@{}}by activity\\ by name\\ by street name\end{tabular} & \begin{tabular}[c]{@{}l@{}l@{}}N [(F \textbar C)][T], [P], SN, SNUM.\\ N [(F \textbar C)][T], [A], SN, SNUM.\\SNUM N [(F \textbar C)][T], [A].\end{tabular} & \begin{tabular}[c]{@{}l@{}l@{}}309\\ 961\\ 0\end{tabular}
\\ \hline
Didot & 1854a & \begin{tabular}[c]{@{}l@{}l@{}}4\\ 3\\ 5\end{tabular} &	\begin{tabular}[c]{@{}l@{}l@{}} by activity \\ by name \\ by street name\end{tabular} & \begin{tabular}[c]{@{}l@{}l@{}} N [(F \textbar C)], [P], SN, SNUM [,T].\\ N [(F \textbar C)][T], [A], SN, SNUM.\\ SNUM N [(F \textbar C)][T], [A].\end{tabular} & \begin{tabular}[c]{@{}l@{}l@{}}106\\ 362\\ 0\end{tabular}
\\ \hline
Bottin (serie 3: 1854-1856) & 1854a & \begin{tabular}[c]{@{}l@{}l@{}}3\\ 2\\ 4\end{tabular} & \begin{tabular}[c]{@{}l@{}l@{}} by activity \\ by name \\ by street name\end{tabular} & \begin{tabular}[c]{@{}l@{}l@{}}N [(F \textbar C)], [P], SN, SNUM [,T].\\ N [(F \textbar C)][T], [A], SN, SNUM.\\ SNUM N [(F \textbar C)], [A] [,T].\end{tabular} & \begin{tabular}[c]{@{}l@{}l@{}}253\\ 633\\ 0\end{tabular}
\\ \hline
Didot-Bottin & 1860a	& 3	& by name & N [(F \textbar C)][T], [A], SN, SNUM. & 378
\\ \hline
Didot-Bottin & 1861a	& 3	& by name & N [(F \textbar C)][T], [A], SN, SNUM. & 402
\\ \hline
\end{tabular}%
}
\caption{The directories used to build our dataset. Letters next to year of publication are used to identify the volume of the directory. \# Col stands for "Number of columns", and \# Ann stands for "Number of annotated entries". Entry structure is described as follows: N - entry name (person or business); F - first name; T - honorary title; C - civility; A - activity; P - precision; SN - street name; SNUM - street number; SEC - section; D - district; CR - cross-reference; bracketed elements are optional.}
\label{tab:directories}
\end{table}

\subsection{Groundtruth data extraction}

To build our ground truth corpus, we annotate a few pages of each of the major directory series for which we have digitised copies, using a tool developed for our project (see Figure \ref{fig:annotator}). First, the image of a given directory page is loaded. Second, entries are automatically detected and boxes are drawn around them as shown in the left panel of the Graphical User Interface (GUI). Then the text of each entry is extracted by an OCR tool and finally named entities are annotated by a NER tool: the results are displayed respectively in the first and second tabs of the right panel of the GUI. The entry detection, OCR and NER tasks are performed automatically. But as they are prone to errors, their results can be corrected manually through the GUI. In the end, the entry bounding boxes, their text, and the named entities are stored in a JSON file.

\begin{figure}[htb!]
	   \center{\includegraphics[width=0.9\textwidth]
	       {./images/Annot2.png}}
	  \caption{\label{fig:annotator} GUI for corpus annotation}
\end{figure}

The annotated named entities are as follows:
\begin{itemize}
    \item PER: a person name or a business name, usually referred to as several person names. Firts names, initials, or civility mentions are included. E.g.: "Alibert (Prosper)", "Allamand frères et Hersent", "Heurtemotte (Vve)", etc.
    \item TITLE: an honorary title, either written or represented by special glyphs. E.g.: "élig.", \scalerel*{\includegraphics{./images/LH.png}}{H} which stand for the Légion d'Honneur, \scalerel*{\includegraphics{./images/GMLondon.png}}{H} which stands for the Great Medal at the London exhibition, .etc.
    \item ACT: the activity of the person (profession or social status) or the type of business. E.g.: "horlogerie", "export en tous genres", "Conseiller d'Etat", "propriétaire", etc.
    \item LOC: a street name or a district name. E.g.: "r. de la Jussienne", "Au Marais", etc.
    \item CARD: a street number, as part of an address. E.g. "16", "5 bis", etc.
    \item FT: a geographic entity type, used to give more details on a location. E.g.: "boutique", "atelier", "fab.", etc.
\end{itemize}

Thus, the extracted corpus contains both the bounding boxes of directory entries, their exact text and the named entities they contain. It can therefore be used to train and improve the entry detection, OCR and NER tools.

\subsection{Assessing the groundtruth quality}
\nathalie{On va utiliser cette sous-section ou pas?}
Our dataset is thus made of excerpts TODO interesting points to detail:

\begin{itemize}
    \item statistics about entry lengths (histogram of number of char / tokens par entry)
    \item entities we consider and how we annotated them (how we assessed gold's quality)
    \item statistics on the distribution of entities across directories and their arrangement within entries
    \item OCR data (automated systems and clean text + how we created the "clean" text + whether/how we assess clean text's quality)
    \item Measure inter-annotator agreement ? 
    \item Explain charset projection
    \item \dots
\end{itemize}
