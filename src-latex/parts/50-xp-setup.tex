% !TeX root = ../main-paper.tex
\section{Experimental Setup}

\subsection{Metrics}
\bertrand{Move after section Cases studies? }
\begin{itemize}
    \item NER quality: Precision, Recall, F1 score (idea: stick with standard spacy/sklearn metrics) \nathalie{Dans la plupart des articles que j'ai lus, c'est toujours une F-mesure qui est utilisée et parfois la précision et le rappel sont ajoutés.}
    \item OCR accuracy: outputs from standard tools (beware of unknown glyphs: do we need any special care?) \nathalie{Pour l'instant, j'ai croisé: CER, WER, Levenshtein et Jaccard. J'ai encore un état de l'art à lire sur le sujet: il y aura probablement d'autres mesures mentionnées.}
    
    \nathalie{Pour la partie Experiment 2: dans la litérature j'ai vu des courbes F-Mesure(CER) ou F-Mesure(WER) et des diagrammes avec F-Mesure calculée sur des sous-ensembles du corpus classés par score de Levenshtein.}
\end{itemize}

\subsection{Cases studies}
\subsubsection{Experiment 1: NER sensibility to the number of training samples}
The first experiment evaluates the performances of the three models on training sets of different sizes. To do so we split the gold reference into a training set, a development set and a test set. The training set is then gradually reduced in size while maintaining the relative frequency of directories within. The training and testing procedure is the same for the three models.

As the form an structure of entries varies across directories collections and through time, the models may learn to overfit on a subset of directories with specific features. To reduce the evaluation bias we start by leaving out 3 directories (1690 entries, ~20\%) from the gold reference to test each model on unseen directories. Then a stratified sampling based on the entry directory name is applied to the remaining set to create a training (6373 entries, ~71\% of the gold reference) and a development set (709 entries, ~8\%). This sampling procedure is a convenient way to shape both sets to reflect the diversity of directories within the gold reference.

To generate smaller training sets we start from the initial training set and attractively split it in half using the same stratified sampling strategy. We stop if there is only one entry left in a directory or if the current train set contains less than 30 entries. Applying this procedure to the initial training set produced 8 training sets containing 49, 99, 199, 398, 796, 1593, 3186 and 6373 entries.

% Move to metrics ?
All metrics are evaluated on the test set, yet the biased model performances on the development sets are add in the paper material as additional information.


\subsubsection{Experiment 2: NER in the presence of noisy OCR texts}

\begin{itemize}
\item At least 1 run per dataset, more if possible.
\item Only use the best model; should be CamemBERT+pretrained
\item Train on reference gold, evaluate on reference gold AND ocr-gold (2 datasets)
\item split train/dev/test the same way than experiment 1. Use the exact same splitting for datasets so the clean test set, pero test set and tesseract test sets contains the same entries.
\end{itemize}
 
 
%\begin{itemize}
%\item All : leave out 3 directories (~20\% of the dataset) as test data. Then apply stratified sampling based on the directory of each entry to split the remaining into a train set (~72\%) and a dev set (~8\%).
%\item Spacy CNN : No limit on epochs, instead use patience 1600.
%\item CamemBERT: 3 epochs.
%\item CamemBERT pretrained: 3 epochs, pretraining on MLM + NSP with ~20000 raw texts extracted with Pero-OCR.
%\end{itemize}



\subsection{Parameters for each method}

\subsection{OCR systems}
\begin{itemize}
    \item original OCR
    \item Tesseract 4
    \item Pero
    \item other?
\end{itemize}

\subsection{Implementation details}
We use Spacy\mcite{spacy} for the CNN implementation
and Huggingface for the BERT 
tok2vec(words embeddings + encoding) + attention layer  +  transition-based model.

\paragraph{Huggingface CamemBERT}
CamemBERT + linear classifier as decision layer



we use spacy  for the CNN implementation and Huggingface for BERT.

will we use FLAIR as well? $\rightarrow$ No