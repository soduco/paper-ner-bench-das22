\documentclass[runningheads,svgnames]{llncs}
%
\usepackage[]{graphicx}
\usepackage{amsmath}
\usepackage{hyperref}
\usepackage{cleveref}
% \usepackage{tikz}
% \usepackage{nth}
% \usepackage{subcaption}
\usepackage{booktabs}
% \usepackage{array}
\usepackage{orcidlink}

% \usetikzlibrary{shadings,decorations,decorations.pathreplacing}
\renewcommand\UrlFont{\color{blue}\rmfamily}

\newcommand{\mcite}[1]{\text{\cite{#1}}}
\newcommand{\mref}[1]{\text{\ref{#1}}}

% \renewcommand{\baselinestretch}{0.97}


\begin{document}
%
\title{A Benchmark of Named Entity Recognition Approaches in Historical Documents\\
Application to 19$^{th}$ Century French Directories%
\thanks{This work was supported by the French National Research Agency:
Project SoDuCo, grant ANR-18-CE38-0013. ???FIXME need acknowledgments for directory sources???}}
%
\titlerunning{A Benchmark of NER Approaches in Historical Documents}
%
\author{LISTE PROVISOIRE PAR ORDRE ALPHA\\
%
Nathalie Abadie\inst{1}\orcidID{0000-1111-2222-3333} \and
Edwin Carlinet\inst{2,3}\orcidID{1111-2222-3333-4444} \and
Joseph Chazalon\inst{3}\orcidID{2222--3333-4444-5555} \and
Bertrand Duménieu\inst{3}\orcidID{2222--3333-4444-5555}}
%
\authorrunning{F. Author et al.}
% First names are abbreviated in the running head.
% If there are more than two authors, 'et al.' is used.
%
\institute{INST, CITY, COUNTRY \and
INST, CITY, COUNTRY\\
\email{example@example.com}\\
\url{https://www.example.com} \and
INST, CITY, COUNTRY\\
\email{example@example.com}}
%
\maketitle              % typeset the header of the contribution
%
\begin{abstract}
The abstract should briefly summarize the contents of the paper in
150--250 words.

\keywords{First keyword  \and Second keyword \and Another keyword.}
\end{abstract}
%
%
%
\section{First Section}
\subsection{A Subsection Sample}
we use spacy \mcite{spacy}

% ---- Bibliography ----
\bibliographystyle{splncs04}
\bibliography{ref}
\end{document}
